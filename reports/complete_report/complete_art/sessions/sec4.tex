\section*{4 详细设计}

% 详细设计这,C++后端,vue前端
\subsection{后端设计}
\subsubsection{激光雷达数据处理}

\subsection{触觉反馈计算模型}

% 触觉反馈设计
系统将激光雷达 360° 扫描范围划分为 8 个等角度扇区,每个扇区对应一路振动马达。对于第 $i$ 个扇区,选取该区域内最近障碍物的距离作为风险距离 $d_i$。设定 $d_0$ 为起振距离,$d_1$ 为满振距离,则归一化风险强度定义为:
\[
x_i = \mathrm{clip}\left( \frac{d_0 - d_i}{d_0 - d_1},\ 0,\ 1 \right)
\]

当障碍物逐渐靠近时,$x_i$ 随距离单调增大,能够较为准确地反映环境风险变化趋势。

为增强近距离障碍物的区分度,系统引入非线性增强函数:
\[
\alpha_i = (x_i)^{\gamma}
\]
其中 $\gamma > 1$。实验中发现,该映射能够有效放大近距离风险变化,使触觉反馈更加敏感。

同时,为避免雷达噪声引起振动强度的快速抖动,引入一阶低通滤波对触觉信号进行时间平滑:
\[
A_i(t) = (1 - \alpha) A_i(t - 1) + \alpha a_i(t)
\]

平滑后的振动强度变化更加连续,显著提升了触觉反馈的舒适性与可感知性。

最终触觉强度 $A_i \in [0,1]$ 被映射为 PWM 占空比。实验中定时器自动重装寄存器设置为 $ARR = 999$,对应比较寄存器值为:
\[
CCR = \left\lfloor A_i \times ARR + \frac{1}{2} \right\rfloor
\]

该映射方式保证了计算端与执行端 PWM 数值范围的一致性,实现了风险强度到物理振动的连续映射。

\subsection{触觉反馈模块设计}

\subsection{前端设计}
\subsubsection{}
