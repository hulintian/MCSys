\section*{第五章 实验结果与分析}

本章围绕系统在实际运行条件下的实验结果展开分析,对激光雷达感知、触觉反馈计算、串口通信与执行控制以及多模态环境理解等关键模块的实验效果进行验证与讨论。

\subsection{系统实验平台与整体运行状态}

实验平台由 TL-3588 处理平台、激光雷达、摄像头以及 STM32F407 控制板构成,系统整体采用“感知—计算—通信—执行”的分层架构。TL-3588 负责环境感知与控制量计算,STM32 负责底层实时执行与触觉输出。

图 \ref{fig:hw_setup} 展示了系统的硬件连接与实验搭建情况。实验过程中各模块运行稳定,串口通信与 PWM 输出未出现异常,系统能够持续稳定工作。

\begin{figure}[H]
  \centering
  \includegraphics[width=0.75\textwidth]{imgs/all_hw.png}
  \caption{系统硬件实验平台与连接示意}
  \label{fig:hw_setup}
\end{figure}

\subsection{激光雷达数据读取与可视化实验结果}

\subsubsection{激光雷达数据采集结果}

实验中,TL-3588 通过串口持续接收激光雷达输出的角度与距离数据。在正常工作状态下,雷达单圈扫描可获得约 1440 个测距点,扫描帧率稳定在 20 FPS 左右。原始数据经过解析后被统一映射至极坐标系,为后续环境建模与风险评估提供基础输入。

实验结果表明,激光雷达数据采集稳定,未出现明显丢帧或数据异常情况,满足实时感知需求。

\subsubsection{前端雷达点云可视化分析}

激光雷达数据在前端页面中以二维极坐标形式进行可视化显示,如图 \ref{fig:lidar_view} 所示。系统以“前、后、左、右”为参考方向对点云进行标注,使环境结构更加直观。

从实验结果可以观察到,当环境中存在障碍物时,点云在对应方向形成明显聚集;障碍物位置发生变化时,点云分布能够实时更新,表明雷达数据处理与前端渲染链路工作正常。

\begin{figure}[H]
  \centering
  \includegraphics[width=0.8\textwidth]{imgs/fd_lidar.png}
  \caption{激光雷达点云前端可视化结果}
  \label{fig:lidar_view}
\end{figure}

\subsection{串口通信与 STM32 执行结果分析}

\subsubsection{串口通信协议实验验证}

系统采用自定义串口数据帧格式进行 tl-3588 与 stm32 之间的数据传输。该协议包含帧头、类型、序号、长度、有效载荷与 crc 校验字段,能够有效支持 8 路 pwm 数据的可靠传输。

实验过程中未出现因粘包、拆包或数据丢失导致的控制异常,验证了通信协议设计的健壮性。

\subsubsection{STM32 接收与 PWM 执行分析}

STM32 端采用 UART + DMA + FreeRTOS 的方式进行数据接收与解析。接收任务在完成协议校验后,将 8 路 PWM 数据写入 TIM3 与 TIM4 的对应比较寄存器,实现多路振动马达的独立控制。

实验结果表明,不同方向障碍物的变化能够实时反映为对应马达振动强度的变化,控制链路响应及时,系统运行稳定。

\subsection{环境场景描述实验结果}

系统通过 OpenCV 获取前方环境的单帧图像,并调用 Qwen 多模态接口生成场景语义描述。图 \ref{fig:camera_scene} 展示了实验过程中获取的环境图像及前端显示界面。

实验结果表明,该方法能够对办公场景中的主要物体与空间关系进行简要描述,为系统提供补充的环境语义信息,验证了多模态感知方案的可行性。

\begin{figure}[H]
  \centering
  \includegraphics[width=0.8\textwidth]{imgs/fd_cam.png}
  \caption{基于摄像头的环境场景获取与描述结果}
  \label{fig:camera_scene}
\end{figure}

\subsection{本章小结}

本章通过实验验证了系统在激光雷达感知、触觉反馈计算、串口通信与执行控制以及多模态环境理解等方面的功能与性能。实验结果表明,系统整体运行稳定,各模块协同工作良好,达到了预期设计目标。
