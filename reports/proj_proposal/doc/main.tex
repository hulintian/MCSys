% !TEX program = xelatex
\documentclass[hyperref,a4paper,UTF8]{ctexart}

\usepackage[left=2.50cm, right=2.50cm, top=2.50cm, bottom=2.50cm]{geometry}

\usepackage[unicode=true,colorlinks,urlcolor=blue,linkcolor=blue,bookmarksnumbered=true]{hyperref}
\usepackage{latexsym,amssymb,amsmath,amsbsy,amsopn,amstext,amsthm,amsxtra,color,bm,calc,ifpdf}
\usepackage{graphicx}
\usepackage{enumerate}
\usepackage{fancyhdr}
\usepackage{listings}
\usepackage{multirow}
\usepackage{makeidx}
\usepackage{xcolor}
\usepackage{fontspec}
\usepackage{subfigure}
\usepackage{hyperref}
\usepackage{pythonhighlight}

\title{空间风险场的信息论触觉编码}

\date{}


\begin{document}
\maketitle

% OutLine
% 开题报告
% 1. 研究现状
% 2. 需求分析
% 3. 设计方案
% 4. 创新点
% 
% IDEA
% ## 绝妙的点子
% 
% - 基于多模态数据融合的智能感知系统
%   - 导盲 -- 不一定导盲,环境感知
%   - 人体,智能感知
% 
% ### 实现方式
% 
% - 激光雷达,感知环境,然后将每个方向的障碍物的距离,映射成振动马达的振动频率
% - 摄像头环境感知

\section{研究背景}

视觉信息在人类感知、定位与导航活动中占据核心地位。然而,根据世界卫生组织(WHO)的统计,全球约有 2.85 亿视障人群,其中重度视力损伤者超过 3900 万。视障群体在日常出行中普遍面临道路障碍识别困难、环境动态变化难以及时感知、人群交互风险高等问题,严重影响自主出行能力与生活质量。传统辅助工具,如白手杖与导盲犬,虽能提供一定程度的物理探测或陪伴支持,但它们的环境感知距离有限、信息维度单一、适应复杂场景能力不足,并难以在动态环境中持续提供高可靠的空间信息。

随着智能传感器、计算平台与人工智能技术的快速发展,可穿戴智能辅助系统逐渐成为视障辅助领域的重要方向。基于 RGB-D 摄像头、深度传感器、激光雷达、IMU 等多模态传感器的装置能够实时构建周围环境的空间信息;而基于振动、声音或骨传导的反馈装置则能够将系统感知的环境信息映射为用户可感知的信号,实现“机器感知 到 人体感知”的信息转换。其中,触觉通道具有不占用听觉、隐私性强、直观性高等优势,被认为是最适合作为视觉替代和辅助导航的交互方式之一。

然而,在现有辅助系统中,“如何高效地将复杂的环境信息转化为触觉刺激”仍然是一个未解决的关键科学问题。环境中的障碍物分布、运动趋势、潜在威胁程度本质上构成一个高维、连续变化的空间风险场(spatial risk field);与此同时,人体皮肤的触觉通道具有带宽有限、噪声较高、可区分模式数量受限等生理约束。现有系统大多采用经验型、分段式或线性映射的方式(如距离越近振动越强、方向对应特定振动位置),难以在有限触觉通道中压缩并传输复杂的空间风险信息,导致信息利用效率低、用户学习成本高、认知负担大,难以在复杂动态场景中实现稳定可靠的导航提示。

另一方面,现代信息论在人机交互领域的应用表明,将触觉提示看作一种“有限信道的信息传输”问题,可以量化触觉通道的有效容量、编码效率与模式可分辨性。这为构建以信息论为基础的触觉编码框架提供了理论基础。然而,目前在视障导航领域,尚缺乏将“空间风险场”作为触觉信源进行建模,并以信息论方法系统优化编码策略的研究。

因此,构建一种面向视障导航的多模态环境感知 → 空间风险场建模 → 信息论触觉编码一体化框架,不仅能够提升触觉提示的表达效率与任务性能,也有望为触觉界面设计、人机融合感知等领域提供新的理论模型,在智能辅助出行、可穿戴感知系统、认知增强等方向具有重要的科学意义与应用价值。

\subsection{国内研究现状}

在国内,近年来涌现出一批面向视障者的多模态感知与可穿戴导航系统,主要集中在工程系统实现与应用推广层面。

上海交通大学顾磊磊团队提出了一套融合柔性电子与 AI 的可穿戴视觉辅助系统\cite{tang2025human}:通过眼镜上的 RGB-D 摄像头和红外探测器采集环境信息,在微型计算平台上进行目标检测与场景理解,将结果以骨传导耳机 + 电子皮肤振动贴片的方式反馈给视障用户,帮助其完成导航和抓取等任务。

在专用触觉导盲设备方面,国内已公开了多项实用新型专利。例如专利 CN211461092U 提出了一种可穿戴导盲系统,包括带路面检测模块的导盲帽和带提醒模块的导盲腰带,通过在帽子上布置传感器采集路面信息,经腰带上的控制模块处理后驱动振动提醒用户避障。

此外,产业界涌现了包括导盲六足机器人、基于四足机器人的“电子导盲犬”等新型辅助系统,这些系统多采用视觉/激光雷达/高精度地图,实现自主导航和避障,再通过语音或简单触觉提示与用户交互。 整体上看,国内在“多模态感知 + 导航控制”方面的工程实现已相当活跃,但在“触觉信道建模与编码优化”的理论层面相对薄弱。

\subsection{国外研究现状}

在智能驾驶和机器人领域,“风险场(risk field)/危险场” 已被广泛用于刻画环境中各方向的碰撞风险强度。Kolekar\cite{KOLEKAR2021103428}等构建了驾驶员风险场(Driver’s Risk Field, DRF),通过车辆间相对位置、速度等参数,计算道路上各位置的风险标量场,并验证该风险度量与驾驶员主观风险感知及驾驶行为之间存在显著相关性。
Luo\cite{doi:10.1177/09544062221085886} 等进一步将“场论”引入车路协同环境,提出智能网联车辆驾驶风险场模型,将交通要素、车辆工况等统一到一个时空连续的风险场中,为自动驾驶决策提供度量基础。
近期工作还基于车道变换等典型场景构建主观驾驶风险预测模型,将认知风险建模为随时间演化的“风险场”信号\cite{10.1109/TITS.2024.3409874}。

这些研究说明:用“场”来抽象空间风险是成熟且有效的做法,但风险场的输出主要用于车辆控制决策或驾驶辅助,并未面向“人体触觉通道”的信息表达进行专门设计,更没有从信息论角度讨论“如何在有限带宽的人体感知通道上最优编码风险场”。

在视障辅助出行领域,国际上已经形成了较为丰富的“可穿戴振动导航/导盲系统”。Dakopoulos 与 Bourbakis 对可穿戴障碍规避电子出行辅助设备(Wearable Obstacle Avoidance Electronic Travel Aids, ETAs)做了系统综述,指出典型系统包括摄像头、深度传感器或超声传感器,以及腰带、背心、手套等触觉反馈终端,通过振动提示障碍物方向与距离\cite{dakopoulos2009wearable}。

在具体系统设计上,Flores 与 Manduchi 提出的 “Vibrotactile Guidance for Wayfinding of Blind Walkers” 将一圈等间距振动致动器布置在腰部,通过振动位置和模式为盲人提供行走方向提示,并在室内走廊等场景中评估导航性能和主观体验。
Salazar-Luces 等人利用固定激光扫描仪构建路径规划,将行人建模为非完整机器人,通过 ROS 局部导航规划得到线速度、角速度,再映射为腰带上的振动模式,引导使用者沿规划路径行走。
Altini 等提出了低成本室内振动导航系统:利用室内定位与蓝牙通信,在腰带振动装置上编码转向与距离信息,帮助视障者完成室内导航任务。

在触觉编码策略方面,Zeng 等工作从“二维触觉词汇(2D tactile vocabulary)”角度研究导航用触觉编码,探讨不同空间振动模式在方向指示上的可区分性与直观性,为构建“触觉导航语言”提供了基础\cite{10.1145/2384916.2384936}。
van Erp 等针对“Obstacle Detection Display for Visually Impaired”提出了系统的实验研究:通过腰带/背心布置多点振动致动器,尝试同时编码障碍的方向、距离和高度,并通过一系列心理物理实验给出设计建议——例如建议单条触觉消息中最多使用两种编码参数(如位置 + 节奏),以避免认知负荷过高\cite{dakopoulos2009wearable}。
Brandebusemeyer 等则从情绪和舒适性角度评估振动腰带在复杂导航场景中的作用,发现佩戴振动腰带能显著提高使用者的主观安全感和信心。

此外,还出现了多种将视觉信息转化为触觉的感觉替代系统。例如 Johnson 与 Higgins 的 Tactile Vision System(TVS)用摄像头采集前方图像,通过腰部 14 个振动马达阵列显示空间结构信息,帮助视障者主动避障。
这些研究表明:周向振动腰带 + 合理的编码策略 能够在不占用视觉和听觉通道的情况下,为用户提供较为精细的方向与障碍信息。

总体来看,现有国内外触觉导盲系统的核心问题在于:
\begin{enumerate}
    \item 大多数仅将障碍信息离散化为“若干方向 + 若干距离等级”,
    \item 触觉编码多靠经验和心理物理试验调整,
    \item 尚未将环境抽象为连续的“空间风险场”,并针对这一高维信息源系统地优化触觉编码方案。
\end{enumerate}

\section{需求分析}

面向视障用户的自主出行辅助系统,需要在复杂室内外环境中稳定工作,并在不增加认知负担的前提下,向用户提供及时、准确且易于理解的环境提示。基于对现有系统的分析以及目标应用场景的需求,本课题将需求分为用户需求、系统功能需求和性能需求三个层次。

\subsection{用户需求分析}

在日常出行过程中,视障用户普遍面临环境不可见、障碍物不可预测、动态目标难以识别等问题,传统白手杖只能提供有限的触碰式反馈,无法实现提前预警,特别是在复杂环境(如校园、街区、公交站、公共建筑)中更易造成潜在风险。因此,用户首先需要一种能够实时提供周围环境信息的辅助方式,包括障碍物的方向、距离、大小、运动趋势以及潜在危害程度,使其能够在行动前做出及时判断。其次,由于视障用户在行走过程中依赖听觉感知环境声源(车辆声、人流声等),辅助系统应尽量避免占用听觉通道,而触觉成为更适合的反馈形式。因此,用户需要一种简洁、直观、低认知负担的触觉提示方式,能够在瞬时内被感知和理解,而不要求复杂的学习成本。再次,系统的提示必须具备稳定性和可预测性,避免因环境噪声、照明变化或动态人群导致输出模式混乱,这直接关系到用户对系统的信任度和使用安全。最后,由于每位用户的触觉敏感度、操作习惯不同,系统还应支持一定程度的个性化调节,使其在使用过程中能够逐渐适配用户的触觉感知特性,提升整体体验。

\subsection{系统功能需求分析}

为了满足视障用户的导航需求,系统在功能层面必须包含从环境感知到触觉反馈的完整信息链路。首先,系统需要具备稳定、连续的多模态环境感知能力,通过 LiDAR、RGB摄像头与 IMU 的融合,实时获取周围 360° 的障碍物分布与动态目标信息。其次,系统必须能够将原始感知数据转化为结构化的空间表达,即构建空间风险场,用于量化不同方向的潜在危险程度,并在必要时进行短时预测,从而为提前提示提供基础。再者,系统需要具备风险场向触觉反馈的映射能力,能够将高维环境信息压缩为有限触觉执行器可表达的振动模式,包括方向、强度、频率与节奏等多参数组合,以保证提示的表达能力。系统还应支持用户自适应功能,根据使用历史与触觉敏感度动态调整编码策略,避免提示过载或过弱。最后,系统还需具备运行管理与异常处理能力,包括任务调度、电源管理、硬件状态监测等,以保证长期运行的稳定性。

\subsection{性能需求分析}

在实际导航场景中,视障用户对系统的性能要求极为严格。首先,系统必须具备高实时性,环境感知、风险场生成和触觉反馈的总时延应控制在 150 毫秒以内,以保证提示在用户行动之前产生效果。其次,系统需要具有高可靠性,尤其是障碍物检测的漏报率和虚警率必须分别保持在较低水平,以防止对用户造成安全隐患。此外,触觉提示的可辨识度是影响系统效果的关键因素,系统输出的振动模式应在短时间内被用户明确区分,其平均辨识率应达到 80\%–90\% 以上,并能够在较少次训练后掌握。系统在长时间运行时还需具备良好的舒适性,包括振动强度控制、皮肤疲劳抑制、温升限制以及轻量化要求,使用户能够在连续佩戴过程中保持舒适。最后,系统需要具备足够的环境适应性,能够在光照变化、遮挡、动态人群、室内外切换等多种场景下保持稳定运行,不因外部环境变化而导致性能不稳定。

\section{总体设计思路}

本设计以“空间风险场的信息论触觉编码”为核心思想,围绕视障用户的实时安全导航需求,构建一套从多模态环境感知到触觉信号输出的完整系统架构。总体思路是:通过多模态传感器获取环境结构与动态信息,将其统一表示为连续的空间风险场;随后在触觉通道带宽有限的约束下,以信息论为基础设计最优触觉编码,将高维风险信息压缩为用户可快速理解的振动模式;最终通过可穿戴触觉阵列输出提示,实现实时导航辅助。整体架构在设计中强调实时性、可解释性、稳定性与用户可用性四个方面。

\subsection{多模态环境感知与融合}

本项目采用激光雷达(LiDAR)与普通 RGB 摄像头作为主要感知硬件,以实现对环境结构和语义信息的全面获取。LiDAR 提供高精度、鲁棒的距离测量能力,能够在 360° 范围内输出稀疏但可靠的障碍物空间分布,适用于构建精确的局部几何地图。RGB 摄像头则提供丰富的语义信息,可用于识别行人、车辆、门框、楼梯、标志物等环境元素,弥补 LiDAR 无法区分物体类别的不足。

为了获得稳定的环境表示,系统通过时间同步与外参标定,将 LiDAR 点云和 RGB 图像统一到同一坐标系,构建“几何 + 语义”的组合式环境模型。具体而言,LiDAR 的距离信息用于精确定义障碍物的方位与大小,而 RGB 图像中的检测结果用于标注障碍物类型、动态属性及行为趋势。通过简单的投影关系或轻量级深度估计模型,系统能够将 RGB 的语义信息与 LiDAR 的距离信息结合,实现对环境的结构化感知,为后续的空间风险场建模提供基础。

\subsection{空间风险场构建与时序预测}

在获得多模态感知信息后,系统采用连续场(field-based)模型将环境表征统一为空间风险场(Spatial Risk Field)。风险场通过对不同障碍物施加风险函数并按方向整合,形成一个描述“在每个方向上可能存在的危险程度”的连续标量函数。

风险场包含两类信息: 静态风险,如墙壁、家具、楼梯等;和动态风险,如行人、车辆等动态目标。 系统将这两类信息融合,通过距离衰减函数、相对速度权重、障碍类别权重等综合构建风险值,使风险场不仅反映障碍物的位置,也反映潜在碰撞的可能性。

此外,系统引入轻量级运动预测模型,根据过去数帧环境状态预测未来的风险场变化,使触觉提示具有前瞻性,避免仅对当前危险做被动反应。例如,当存在高速靠近的动态物体时,系统能够提前发出警告,从而显著提升使用者的安全性。

\subsection{信息论驱动的触觉编码设计}

触觉反馈作为系统的核心输出方式,需要在人体皮肤带宽有限、噪声干扰较多的情况下,最大限度传达环境中的高维风险信息。因此,本课题采用信息论视角设计风险场到触觉信号的压缩映射,确保编码具有最大可区分度与最小信息损失。

在触觉通道容量约束下,系统通过最大化互信息的方式构建编码方案,使振动模式最能代表风险场中对导航重要的信息。映射方式包括:
\begin{enumerate}
    \item 方向编码如:马达位置或多点组合
    \item 强度编码如:风险大小
    \item 频率编码如:危险等级或动态变化速度
    \item 节奏编码如:风险变化趋势,如即将碰撞
\end{enumerate}
通过对这些参数进行组合优化,系统生成最适合用户的触觉模式表,提高提示的清晰性和稳定性。

\subsection{可穿戴触觉反馈系统实现}

根据编码结果,系统通过可穿戴的环形振动阵列将触觉信号输出到用户腰部。该阵列由多个等间距马达组成,能够在 360° 范围内提供方向化的振动提示。硬件采用串口进行通信,驱动控制器支持精确调节每个马达的振动频率、幅度与节奏,以保证编码的准确呈现。
保证用户在长时间导航任务中依然能够获得稳定、舒适的触觉体验。

\section{创新}

结合前述国内外研究现状可以看出,目前的触觉导航技术仍存在显著不足。现有系统普遍采用基于经验规则的触觉编码方式,如以“距离对应振动强度”“方向对应马达位置”等人为设定的映射规则来表达障碍信息。这类方法缺乏统一的理论基础,难以有效压缩与表达复杂的环境风险信息,同时无法根据环境变化或用户差异进行动态调整。此外,国外触觉研究虽然提出了若干有效的心理物理原则,如 van Erp 关于方向提示和参数可辨性的成果,但其方法依旧属于经验型编码,无法满足复杂动态导航场景下的高效传输需求。而另一方面,自动驾驶和机器人领域中广泛应用的“空间风险场”理论已经证明了其在环境危险度建模方面的优势,但现有研究并没有将其作为信息源引入触觉导航系统,更未与人体触觉通道的生理限制相结合。

针对这些研究空白,本项目提出了三个层面的创新。首先在理论方面,本课题首次将“空间风险场”与“信息论触觉信道”统一建模,将环境中障碍物的距离、方向、速度、类别等多源信息整合为连续的空间风险分布;同时将人体触觉通道视为一个带宽有限、存在噪声的通信信道,从而构建起“环境 → 风险场 → 触觉编码 → 用户动作”的完整信息流模型,为触觉导航系统提供了可量化、可解释的理论框架。相比于以往基于经验的编码策略,这种模型化方式能够从根本上解释触觉提示的有效性,并为系统最优化设计提供数学支撑。

其次,在编码设计方面,本课题提出了一种基于互信息最大化的触觉编码方法,将风险场到触觉模式的映射视为典型的信息论优化问题。在触觉带宽有限和感知混淆不可避免的前提下,系统通过最大化风险场与触觉信号之间的互信息,自动搜索最优的振动编码组合,包括马达位置、幅度、频率、节奏以及多马达组合模式,使编码在生理可辨范围内尽可能保留最多的环境风险信息。与传统仅表达障碍物方向与粗粒度距离的编码方式不同,本课题提出的编码策略能够进一步表达诸如风险上升速度、即将发生碰撞的时间窗口、动态目标接近趋势等关键安全信息,显著提升触觉提示的表达能力。

最后,在系统层面,本课题引入了多模态风险预测与用户自适应反馈机制,构建了一个具有前瞻性和个性化能力的可穿戴触觉导航平台。通过 LiDAR 与 RGB 摄像头融合构建时序风险场,系统能够预测未来短时间内的潜在风险,实现提前提示,从而显著提高实际避障表现。同时,系统根据用户的误判记录、触觉敏感度及操作特征动态调整编码参数,形成随用户使用而不断优化的“学习型触觉提示机制”,克服了现有系统普遍缺乏个体适配和闭环优化的问题。整体而言,该系统不仅具备理论最优性,也具备实际可用性,为触觉导航技术提供了新的范式。

\section{实验设计}

本课题的实验设计围绕“触觉编码的有效性”和“实际导航表现”两个核心问题展开,旨在验证所提出的基于空间风险场的信息论触觉编码方法相对于传统经验式编码方案的优越性。整体上,实验分为三个层次:首先通过可辨识性实验验证不同编码方案在触觉模式区分上的差异;其次通过导航与避障实验评估各方案在实际任务中的安全性与效率;最后通过主观问卷与负荷评估分析用户体验与认知压力。所有实验均采用对比设计,引入多个具有代表性的基线方法作为比较对象,并从客观性能指标与主观感受两个维度对系统进行综合评价。

\subsection{比较对象} 

为了公平、系统地评估本课题方法的效果,实验中将选取三类具有代表性的对比方案作为基线。第一类是传统线性映射触觉编码方案,即将障碍物的最近距离线性映射为振动强度或频率,将方向直接对应为特定马达位置,只提示最近、最大障碍。这类方案在现有振动腰带、电子盲杖等产品中较为常见,能够代表“工程实践中普遍采用的经验式做法”。第二类是参考 van Erp 等研究提出的位置 + 节奏编码方案,即限制编码参数为触觉心理物理研究中验证较好的组合,例如只使用振动位置表示方向、振动节奏表示风险等级或距离,并通过过滤规则限制同时提示的障碍物数量。该方案代表了“基于人因实验优化过的经验式编码方法”,是当前文献中较为成熟的一类设计。第三类是语音导航方案,系统通过耳机或骨传导设备输出如“向左微调”“前方两米有障碍”等语音提示,触觉仅用于少量简单提醒。该方案体现了现实中许多依赖语音提示的辅助系统的典型做法,可用于对比“纯听觉导航”与“以触觉为主”的导航效果和认知负荷差异。

在上述比较对象的基础上,本课题提出的信息论优化触觉编码方案作为实验组:该方案在空间风险场建模的基础上,通过互信息最大化设计振动模式,将方向、危险程度、风险变化趋势等信息压缩到有限触觉通道中,并支持针对不同用户的个性化参数调整。通过与传统线性映射、van Erp 风格编码以及语音导航方案的对比,可以从“未经优化的经验式编码—基于经验的人因优化—基于信息论的理论优化”这一轴线,系统地展示本课题方法的优势。

\subsection{实验一:触觉编码可辨识性实验} 

第一类实验旨在在控制条件下评估不同触觉编码方案的可辨识性和反应速度。实验参与者佩戴同一套振动腰带,在视线被遮挡的情况下完成触觉模式识别任务。系统随机呈现若干方向与风险等级组合的振动模式,参与者需在限定时间内判断振动对应的方向(例如 8 个或 12 个方位)、危险等级或风险变化趋势。每种编码方案(传统线性映射、van Erp 编码、本课题信息论优化编码)采用相同数量的模式集合和相同的试次数,以保证对比公平。

在该实验中,记录的主要数据包括:模式识别的正确率、平均反应时间以及不同模式之间的混淆矩阵。正确率和反应时间反映了编码直观性与使用效率,混淆矩阵则可用于进一步计算不同编码方案在“触觉信道”上的有效互信息,从而量化各方案的信息传输效率。通过对比三类方案在这些指标上的表现,可以验证本课题所提出的信息论优化编码是否在“同等复杂度下提供更高可辨识度”和“同等信息量下带来更低认知负担”。

\subsection{实验二:导航与避障性能实验}

第二类实验在更接近实际应用的场景中进行,旨在评估不同编码方案在行走导航与避障任务中的安全性和效率。实验可在室内走廊、简单障碍场地以及模拟现实环境(如校园一段步行路径)中开展。参与者在佩戴振动腰带并遮挡视线的前提下,根据系统提供的触觉或语音提示完成从起点到终点的行走任务。实验采用交叉设计:同一参与者在不同日期或不同轮次下分别体验传统线性映射方案、van Erp 编码方案、语音导航方案以及本课题的信息论优化方案,路径设置尽可能保持等价难度。

在此类实验中,重点记录的客观指标包括:完成任务所需时间、行走路径长度、路径偏离程度、最小安全距离(与障碍物的最近距离)、碰撞次数(包括轻微触碰与严重碰撞)、紧急制动或停顿次数等。这些指标综合反映了系统在实际导航中的引导能力和安全裕度。本课题预期,信息论优化编码方案能够在碰撞次数、最小安全距离和路径偏离等关键安全指标上明显优于传统方案,同时在任务时间上不显著增加,甚至具有一定优化效果。通过这些数据,可以直接将“理论上信息传输效率更高的编码”与“实际避障表现更好”建立联系。

\subsection{实验三:主观认知负荷与用户体验评估}

第三类实验聚焦于用户的主观感受与使用体验,补充客观性能指标难以反映的维度。在完成上述可辨识性与导航实验后,参与者将依据不同编码方案的使用体验填写标准化量表和主观问卷。可以选用 NASA-TLX 量表评估任务负荷,从心理需求、时间压力、努力程度、挫折感等维度量化不同方案对用户认知负荷的影响;同时使用系统可用性量表(SUS)或自定义问卷评估系统的易用性、学习难度、舒适性、信任感及整体偏好等。

通过对比各方案在主观负荷和用户偏好上的评分,可以判断信息论优化编码是否在“信息量更多”的前提下未显著增加,甚至降低了用户的主观负担;同时也可以分析参与者是否更愿意在长期使用中选择本课题所提出的方案。主观结果与客观导航表现的结合,将为本课题方法的实用性提供更完整的支持。

\subsection{评估指标与数据分析方法}

综合而言,本课题的评估指标可分为三大类:编码层面指标、任务性能指标和主观体验指标。编码层面主要关注触觉模式的识别正确率、反应时间以及由混淆矩阵计算得到的互信息与有效信道容量;任务性能层面关注完成时间、路径偏差、最小安全距离、碰撞次数和紧急停顿次数等;主观体验层面则通过 NASA-TLX、SUS 等量表反映用户的负荷和可用性感受。在数据分析上,将采用方差分析(ANOVA)或非参数检验对不同方案的指标进行显著性检验,并在必要时结合事后多重比较分析各方案之间的差异来源。通过上述实验设计与评估体系,可以从“理论信息量—触觉可辨性—导航安全性—用户体验”四个层面系统验证本课题提出方法的有效性与优越性。

\bibliographystyle{plain}
\bibliography{refs}

\end{document}

% 就拖
% 想做啥
% 列出几个点
% lidar & camera 环境感知
% 触觉反馈 --> 更自然的
% 室内环境导航?室内地图?和环境建模 --> 室内环境?每人做标记,要根据室内地图和摄像头来确定自己的位置二者组合使用的意义
