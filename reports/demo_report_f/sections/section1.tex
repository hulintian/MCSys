\section{Laser-to-Haptic Mapping Model}

\subsection{Laser Scan Representation}

The 2D laser scanner outputs a sequence of $N=1440$ measurement points per revolution.  
Each measurement is represented as
\begin{equation}
\mathbf{p}_i = \bigl( \theta_i^{\mathrm{raw}},\, d_i^{\mathrm{raw}} \bigr), 
\quad i = 1,\dots,1440 ,
\end{equation}
where $\theta_i^{\mathrm{raw}}$ and $d_i^{\mathrm{raw}}$ denote the raw angle and distance values, respectively.

The corresponding physical quantities are defined as
\begin{equation}
\theta_i = \frac{\theta_i^{\mathrm{raw}}}{100} \;\; (\mathrm{deg}), 
\qquad
d_i = \frac{d_i^{\mathrm{raw}}}{100} \;\; (\mathrm{m}).
\end{equation}

By sensor convention, $\theta_i = 0^\circ$ corresponds to the backward direction of the user.

---

\subsection{Angular Coordinate Normalization}

For intuitive spatial perception, the angular reference is shifted such that the forward direction corresponds to $0^\circ$:
\begin{equation}
\theta_i' = \operatorname{mod}\!\left( \theta_i - 180^\circ,\, 360^\circ \right),
\label{eq:angle_shift}
\end{equation}
where $\operatorname{mod}(\cdot,360^\circ)$ maps the angle into the interval $[0,360^\circ)$.

---

\subsection{Field-of-View Selection}

Only the forward $270^\circ$ field of view is considered for haptic feedback:
\begin{equation}
\theta_i' \in [-135^\circ,\,135^\circ],
\end{equation}
while measurements outside this interval are discarded.

---

\subsection{Angular Sectorization}

The effective angular range is uniformly divided into $M=8$ sectors, corresponding to eight linear vibration actuators.  
Each sector covers an angular width of
\begin{equation}
\Delta\theta = \frac{360^\circ}{M} = 45^\circ .
\end{equation}

The sector index associated with measurement $i$ is computed as
\begin{equation}
k(i) = \left\lfloor \frac{\theta_i' + 180^\circ}{\Delta\theta} \right\rfloor,
\quad k \in \{0,1,\dots,7\}.
\label{eq:sector_index}
\end{equation}

---

\subsection{Risk Distance Extraction}

For each sector $k$, a set of valid distance measurements is defined as
\begin{equation}
\mathcal{D}_k =
\left\{
d_i \;\middle|\;
k(i)=k,\;
0 < d_i \le d_{\max}
\right\},
\end{equation}
where $d_{\max}$ denotes the maximum reliable sensing range.

The representative distance of sector $k$ is chosen as the minimum distance,
\begin{equation}
d_k =
\begin{cases}
\min \mathcal{D}_k, & \mathcal{D}_k \neq \varnothing, \\
d_0, & \mathcal{D}_k = \varnothing,
\end{cases}
\label{eq:min_distance}
\end{equation}
which reflects the most conservative (highest-risk) obstacle estimate.

---

\subsection{Distance-to-Intensity Mapping}

Let $d_0$ denote the distance at which vibration starts, and $d_1$ the distance corresponding to maximum vibration intensity ($d_1 < d_0$).  
A normalized risk intensity is computed as
\begin{equation}
x_k =
\operatorname{clip}
\left(
\frac{d_0 - d_k}{d_0 - d_1},\;
0,\;
1
\right),
\label{eq:normalized_risk}
\end{equation}
where $\operatorname{clip}(x,0,1)$ limits the value to the interval $[0,1]$.

To enhance sensitivity to near-field obstacles, a nonlinear shaping function is applied:
\begin{equation}
a_k = x_k^{\gamma},
\qquad \gamma > 1 .
\label{eq:nonlinear_mapping}
\end{equation}

---

\subsection{Temporal Smoothing}

To suppress rapid fluctuations, an exponential moving average (EMA) filter is employed:
\begin{equation}
A_k(t) = (1-\alpha)\,A_k(t-1) + \alpha\,a_k(t),
\qquad 0<\alpha<1 .
\label{eq:ema}
\end{equation}

A dead-zone threshold $A_{\min}$ is further applied to eliminate low-amplitude vibrations:
\begin{equation}
\tilde{A}_k =
\begin{cases}
0, & A_k < A_{\min}, \\
A_k, & A_k \ge A_{\min}.
\end{cases}
\label{eq:deadzone}
\end{equation}

---

\subsection{PWM Output Mapping}

Finally, the filtered intensity is mapped to a pulse-width modulation (PWM) duty cycle.  
Assuming a timer resolution of $\mathrm{PWM}_{\max}=999$, the output for sector $k$ is given by
\begin{equation}
\mathrm{PWM}_k =
\left\lfloor
\tilde{A}_k \cdot \mathrm{PWM}_{\max}
+ \tfrac{1}{2}
\right\rfloor,
\quad k=0,\dots,7 .
\label{eq:pwm_output}
\end{equation}

---

\subsection{Overall Mapping Summary}

Combining the above steps, the laser-to-haptic mapping for the $k$-th actuator can be summarized as
\begin{equation}
\mathrm{PWM}_k
=
\left\lfloor
\operatorname{EMA}
\left(
\left[
\operatorname{clip}
\left(
\frac{d_0 - \min \mathcal{D}_k}{d_0 - d_1},
0,1
\right)
\right]^{\gamma}
\right)
\cdot 999
\right\rceil .
\end{equation}
